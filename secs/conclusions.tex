\section{Conclusions}
As shown in the above example studies and research, wearable sensors are adding a new layer of possibilities to correlate discrete event data with continuous physiological data in multiple domains~\cite{alghowinem2016multimodal,huckins2019fusing,mehrotra2016towards,saeb2015mobile,wahle2016mobile}. The combination of continuous and discrete data affords us to investigate data in various perspectives. Fortunately, wearables such as the Fitbits are becoming more popular not only because it is affordable and the data collection is easy, but also because these wearable platforms have the infrastructure to pre-process all the collected data into a human-digestible form. \textit{Now the question is how can we deliver this elegant, pre-processing to education as well}. 

To explore this challenge, we contextualized the merging of continuous and discrete data in a formal educational setting where students' heart rates were collected while they took a standardized test. The continuous heart rate data paired with discrete standardized test performance data was fed into a Same-Size-K-Means algorithm for group formations that we provided for the instructors. The instructors then used that information to finalize group formations with their nuanced knowledge about their students. Through this case study, we learned that the marriage of continuous and discrete data can help draw a picture of the overall students' state of mind, which in turn can help instructors make informed data driven personalized educational decisions in a learning environment.


\textcolor{red}{And with this richer narrative created through multiple data streams with the help of ML tools, specific group formation graphs, for example, can be generated for instructor's use. Our user survey tested whether such ML techniques could work as an aid for teachers in the classrooms and whether more tools can be utilized to further encourage DDDM. SURVEY results..blah blah}

This fine-grained data that was not captured by the test scores alone --- a student may struggle with a section because they lack the knowledge, because they are nervous, or a combination of both factors --- allowed us to take the first steps into \textit{moving away from differentiated instruction to truly personalized instruction} with the aid of classroom wearable technologies and data driven decision making.
