\section{Evaluation Study}

Refer to the following Figure as an example of what we can do. see Figure~\ref{fig:screenshots}

\ref{fig:hearrate_graphs}

\subsection{Participants}
We will recruit 40 users from Amazon Mechanical Turk (AMT) to participate in this phase of the study. Participants will meet the following inclusion criteria: 1) Own an Android mobile device, 2) Be above 18 years of age (self-reported on AMT).
Participants will be asked to download an APK file where they will complete the pre-task survey, the nutritional task of the simplified game, and the post-task survey. 

The participants for Phase 2 will be divided into two groups based on their highest level of educational achievement. We will aim to recruit users of lower education achievement using Amazon Mechanical Turk. Participants will meet the following inclusion criteria: 1) Highest educational achievement completed as High school/GED or lower, 2) Own an Android mobile device, 3) Are above 18. 
To reach users with a reported highest educational achievement of above high school/GED, we will recruit using various forms of social media platforms.

Meals for Monsters (M4M) game for learner-centered and crowdsourced nutrition education.

The study has been approved by the Internal Review Board of two east coast universities collaborating on this project.

\subsection{Research Questions and Hypotheses}

\begin{enumerate}
    \item Does playing the Meal for Monsters game help users better assess the ``right'' in-the-wild meals to meet specific macronutrient nutritional goals?
    \item Does player-avatar identification in the Meals for Monsters game relate to user learning and experience?
    \item Do users utilize the crowdsourced intelligence or community's feedback when making meal choices?
\end{enumerate}

The hypotheses are:
\begin{itemize}
    \item \textbf{H1}: Meal for Monsters (a pet avatar health game with crowdsourced meal photographs) will help users learn to identify the ``right'' in-the-wild meals for a specific nutritional goal (e.g., reduction of fat, increase in carbohydrates) pre to post-test.
    \item \textbf{H2}: Players who choose pet avatars with a health goal that is personally meaningful in Meals for Monsters are more likely to perform better (i.e., identify the right meal photographs for a specific nutritional goal) pre to post test and show enthusiasm towards the game. 
    \item \textbf{H3}: Over time players will consider the community or crowdsourced wisdom on choosing the ``right'' meals for a specific macronutrient nutritional goal.

\end{itemize}

\subsection{Study Design}
The research study will be divided into \textbf{two phases}, with each phase composed of \textbf{Preparation, Data Collection,} and \textbf{Analysis.}  

\subsection{Phase 1}
The three primary goals of Phase 1 are to 1) collect user-generated responses to be presented within the `Meals for Monsters' (M4M) game in future phases, 2) provide initial insights on the prior knowledge and nutritional literacy of lay users, and 3) assess the user experience and data collection methods of the game. 
To accomplish these goals, Phase 1 user participation will consist of both a pre-task and post-task survey, and a simplified version of the more complete game to be used in Phase 2.

\subsubsection{Preparation}
First, we will create the pre- and post-task survey questions regarding prior knowledge as well as the evaluations of meal photographs to see if learning occurred pre-to-post gameplay. Additionally, the simple version of M4M will be designed and developed with four nutritional goals and pet avatars.
We will also create the game to collect detailed logs of users' interactions (i.e., clicks, time-stamps) to assist in identifying user behavior and level of game interaction.

% In the pre-task survey, users will be asked basic demographic questions along with 12 questions in which they are tasked with identifying which of two meal options best fits a given macronutrient content goal (e.g., which meal photograph is higher in carbohydrates). 
% In these 12 questions, the participants will not receive feedback on the accuracy of their responses. 
% The 12 questions will be made up of 4 questions for each macronutrient (carbs, fat, fiber). 
% Two of those four questions will contain meals that will be used in the game and the post-task survey, while the other two will be only used in the pre-task and post-task surveys.

% The post-task survey is similar to the pre-task survey in that it will be made up of two parts: 18 questions where the user is asked to identify which of two meal options best fit a given macronutrient content goal, and follow-up questions related to their experience in the study. 
% The 18 questions consist of 6 questions for each of the three macronutrients, with 2 questions containing meals repeated from only the pre-task survey, 2 questions with meals repeated in both the pre-task survey and game, and 2 questions containing meals that are new to the user. 

For the main task in the game, we will create a set of five rounds for each nutritional goal. 
The four nutritional goals are: 1) choose the meal with the least amount of carbs, 2) choose the meal with least amount of fat content, 3) choose the meal with the most amount of carbs, and 4) choose the meal with the most amount of fiber. 

Each of the five rounds will consist of four ``in-the-wild" meal photographs. 
The ``in-the-wild" images will be gathered from a set used in the PI's prior crowdsourcing in nutrition studies~\cite{desai2019personal,mitchell2019wish} for which the expert nutritional assessment provided by a professional dietitian is available.  
%(MENTION THAT THE IMAGES ARE FROM MEALYZER AND THEN REFERENCE GLUCOGOAILIE OR GLUCORACLE STUDY -- so that we can say they are from our previous studies)

We will filter the images based on their resolution quality and content clarity, so that users will be able to easily identify the components of the meals.
We will create the rounds so that each round consists of one best choice, one second best choice, and two less desirable choices, ranked as such by their macronutrient content assessment. 
We will then test the accuracy of the meal choice rankings by circulating all rounds to nutrition experts. 
We can then adjust the rounds as necessary to achieve above 50\% expert agreement for all rounds.  

\subsection{Data Collection}

In Phase 1, every user will be assigned one of the four pet monsters, each having their own nutritional goal. 
The user will be introduced to their monster avatar and nutritional goal before proceeding to the five rounds. 
There will be three steps for the user in each of the five rounds. 
First, the user will be presented with the four options of ``in-the-wild" meal photographs, accompanied with brief descriptions of the contents of each meal, where they will select the meal that they believe best fits their pet monster’s nutritional goal. 
Next, the user will be asked to provide a short description of why they selected that particular option. 
Finally, after the user submits their reasoning, the game will show them whether the meal option they selected was correct, and if they were incorrect, which meal option was the best choice. 
Each user will repeat these three steps for all five rounds with the same pet monster and its nutritional goal. 

\subsubsection{Analysis}

Phase 1 will act as a method of data collection for use in Phase 2, as well as an opportunity for the research team to evaluate possible changes and improvements to the study plan and opportunities for data analysis.

\subsection{Phase 2}

The game in Phase 2 differs from the previous phase with the addition of three key features.

\subsubsection*{Key Feature 1: The Ability for the User to Select a Pet Monster Avatar}
When users choose their own monster, and therefore nutritional goal, they are able to establish a connection to their pet monster avatar. We will also ask in the post-survey why the players chose their pet monster avatar with its specific health goal.  

\subsubsection*{Key Feature 2: Viewing Crowdsourced Intelligence}
The inclusion of a community board will allow users to view other users' responses to assist them in deciding which meal best fits the chosen nutritional goal. 
This community board will display the percentage of users that selected each of the four meal options as well as a number of user-generated reasons as to why other users chose that meal option. 
The inclusion of the community board will allow us to analyze how users decisions might be influenced by the (un)popular opinion and reasonings of other users. 
We anticipate that in scenarios where the user is swayed by the community board and then shown to be incorrect, we will see a reluctancy in future rounds to trust the opinions of others over their own instincts. 
However, this will most likely change when the community board helps the user find the correct answer, in which case we anticipate that it will be common to see the user trusting the opinions of others in the future.
  
\subsubsection*{Key Feature 3: A Lightweight Approach with Gamification Features}
There will be three significant game components in the M4M game. 
These gamification features provide a lightweight environment for users to gain exposure and knowledge about nutrition components of different meals.
First, after the user selects a monster avatar and its nutritional goal, they will have the opportunity to name their monster. 
This custom name will be used throughout the game, and allows the user to further develop a connection with their pet monster avatar. 
The second gamification feature will be the changing condition of their pet monster in reaction to the user's meal choices. 
Once the user has viewed the community board and submitted their final meal choice, they will be able to watch their pet monster avatar react to the meal they have chosen to feed them. 
If the pet monster is fed the best choice meal for their nutritional goal, the user will see their pet monster's physical state improve in a short animated morph.
If the user feeds their pet monster the worst meal option, or the second to worst, they will see their pet monster's physical state degrade. 
If the monster is given the second best meal option, their condition neither improves or worsens. 
This visual feedback allows the user to see the impact of their choices for their pet monster, as well as motivate them to provide the best options in the future.
The third game component will be the chance for users to earn and choose accessories for their pet monster. 
If within the five rounds of game play a user answers four or more questions with the best meal option, the user will be able to choose one of four accessories to award their pet monster.

\subsubsection{Preparation}

To prepare for Phase 2, the three key features will be programmed into the M4M game.
We will also filter the user-generated responses from Phase 1 for relevancy and grammatical accuracy. 
The filtered responses will then be used to create the community board for users to view in the game. 
The community board will also feature the calculated percentages of Phase 1 users that selected each meal choice per round.
Along with the creation of the community board, we will amend additional questions to the post-task survey that relate to the added features of the game. 
Since users will be able to select their own monster avatar in this phase, we will ask questions regarding the reason why they chose their avatar. 
We will also ask participants a series of relevant questions from the Player-Avatar Identification Scales~\cite{li2013player} using a 5-point Likert scale. 

\subsubsection{Data Collection}


In Phase 2, users now have the ability to view each of the four pet monster avatars and their nutritional goals before selecting one to proceed with and care for during the game. The user will also have the opportunity to give their monster a custom name. 
Then, there will be five steps for the user in each of the five rounds. 
First, after the user is presented with the four options of in-the-wild meal photographs, accompanied with brief descriptions of the contents of each meal, they will select the meal that they believe best fits their monster’s nutritional goal.  
Second, the user will then be asked to provide a short description of why they selected that particular option. 
Third, the user will be shown the community board of crowdsourced reasons why other users selected one of the four meal options. With this new information, the user will have the option to keep their original meal selection, or switch to one of the other meals.
Fourth, the user will be asked to provide a short description of why they selected that meal option.
Last, after they submit their reasoning, the game will show the user whether the meal option they selected was correct, and if they were incorrect, which meal option was the best choice. 
As described earlier in Key Feature 3, depending on the how successful the user has been, the results also include either a short animated morph of the pet monster reacting to the meal the user fed them or the ability for the user to choose an accessory for their pet monster.
Each user will repeat these five steps for all five rounds with the same pet monster and nutritional goal.

\subsection{Analysis}
We will use a combination of qualitative and quantitative methods to analyze the data collected during the proposed study. For the quantitative measures collected during the study (results of their scores on different questionnaires and accuracy of their nutritional assessment), we will use appropriate statistical tests to examine the differences in means for the measures of interest.




