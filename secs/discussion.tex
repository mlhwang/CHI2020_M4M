\section{Discussion}
\subsection{Threats to Validity}
It is important to note several areas in which the validity of this work may be scrutinized. One area has to do with the validity of the heart rate data collected. Are the heart rate readings captured by wrist-worn wearables accurate in their representation of a person's true heart rates? This is an important question as previous research has found variable accuracy among wrist-worn heart rate monitors \cite{wang2017}. As a point of reference, researchers have argued that a heart rate monitor is considered valid if a correlation of $\geq$ 0.90 and a SEE at $\leq$ 5 beats per minute is achieved between the recorded heart rate and a valid companion, such as an electrocardiogram \cite{terbizan2002,leger1988}. In the work presented here, however, no secondary measure was used-- for practical and logistical reasons-- to validate the heart rate readings captured by the Fitbits. Therefore the heart rate readings recorded by the Fitbit devices are assumed to be valid but have not been cross-checked.  

Another area of this work that raises a validity question is the temporal alignment of the continuous heart rate data and the discrete test-item data. As mentioned, the wearable devices captured heart rate readings continuously. However, in a test taking context, there is no way to mark-- in the stream of continuous heart rate data-- precisely when a student begins engaging with the first item of an exam. This limitation makes it difficult to know for when and how heart rate begins being impacted by the test-taking activity. Our program aligned the heart rate data and the test-item data only when the students entered their answer to the first test item. Therefore how long they thought about the first test item and how that affected their heart rate is not completely aligned. Future work in this vein should use a decisive indicator such as a "Start Now" button press to capture the precise time students begin the test taking experience. Even with such a mechanism, true temporal alignment between different systems may be impossible. Researchers in this area should contemplate what degree of temporal alignment is required to ensure an acceptable degree of correspondence between continuous and discrete data streams recorded by different systems.

\subsection{Future Work}

In future work, we may consider tracking galvanic skin responses (GSR). Some research has shown that GSRs can be a better measure for anxiety~\cite{wang2017} and as such might provide new insights to instructors. However, the measurement techniques of GSRs are typically more invasive than heart-rate, requiring electrodes to be strapped to the fingertips of users. This may present an issue in an educational context, especially with younger students.

\textcolor{red}{Tools for teachers?? NOW WHAT?????}