\section{Introduction}

Unhealthy eating contributes to the obesity epidemic in the United States, which affects 12 million American adolescents and 38\% of adults \cite{trustforamericashealth,cdc2015,cdc2020}. The estimated annual medical cost of obesity exceeded \$147 billion dollars in 2008 [PLACE UPDATED REF]~\cite{ }. There are many reasons why people eat unhealthfully; one of such reasons commonly highlighted by previous research is low nutritional literacy, and, specifically, limited ability to accurately estimate nutritional composition of a given meal \cite{kindig2004health}. 

Nutritional literacy and the ability to estimate nutrition in meals is particularly important for self-management of chronic conditions such as diabetes, kidney disease, and cardiovascular diseases. Managing these conditions often requires dietary restrictions, for example of salt, proteins, or carbohydrates, and other nutrients. However, in order to carry out these restrictions, an individual needs to be able to identify and estimate these nutrients in foods. In the context of diabetes self-management, the ability to estimate inclusion of different macronutrients, particularly carbohydrates (sometimes referred to as ``carb counting"), in a meal can directly impact glycemic control and one's wellbeing~\cite{schillinger2002association,sheard2004dietary}.

At the same time, previous studies have shown that estimating nutrition in meals presents considerable challenges. For example, low-literacy populations often have difficulty interpreting Nutrition Fact Labels often leading to miscalculation of the amount of consumed nutrients \cite{chaudry2013formative,huizinga2009literacy,rothman2006patient}. Even professional dietitians can have difficulty making accurate nutritional estimations. Expert dietitians tend to underestimate portions, and therefore calories, of large meals~\cite{chandon2007obesity}. Previous research demonstrated that healthy eaters underestimate calories more than unhealthy eaters  \cite{chandon2007biasing}. Complex meals with many components and ingredients add another layer of difficulty with nutritional assessment, which is why experts in nutrition as well as highly literate populations also struggle with nutritional estimations just like low literate populations \cite{ }. %%% from UMAP paper 

% CONSEQUENCES OF not being able to estimate nutrition: 
% Studies have also shown that there is a strong correlation between low health literacy and low Health Eating Index scores, high sugar-sweetened beverage consumption, and high risk for disease and disability \cite{berkman2004literacy,zoellner2011health}. 

%%%%%%%%%%%%%%%%%%%%%%%%%%%%%%%%%%%%%%%% INTRODUCE LATER:
% \textit{\textbf{In this research, we developed a new solution to help people quickly estimate carbohydrates in meal photos by providing compositionally similar, user-generated, comparison meal photos and their evaluations, as benchmarks for estimating nutrition in their own meal.}}

%Unfortunately, those who struggle with low health literacy tend to eat unhealthfully which in turn increases the likelihood of obtaining a chronic disease such as diabetes. Ironically, those with diabetes that are likely to have low nutritional literacy are usually most in need for a strict low-carb diet, for example. 


%%%%%%%%%%%%%%%%%%%%%%%%%%%%%

% Unknown quantities, e.g. in restaurants or
% bakeries, or incomplete information on nutrition panels
% on packed food, complicate this investigation. The young
% people with diabetes and their family members might use
% the information on the products or look up nutrition
% information in books or on the Internet.

The challenge of nutritional estimation often arises in the context of diet tracking. Many individuals track their food and dietary habits for weight loss, sports training, management of chronic diseases, or food allergies. Previous research argued that in some cases, simple visual records of one's meals may be sufficient to raise individuals' awareness of their eating habits \cite{}. However, in many cases, availability of nutritional information is critical to individuals' ability to notice important trends, to track changes in their diets over time, and to monitor their progress towards their nutritional goals. In the context of chronic disease self-management, these goals often include restrictions in macronutrients. 


%Many people track their food and dietary habits for weight loss, sports training, management of chronic diseases, or food allergies. However, due to the difficulty in nutritional estimation (NE) such as carb counting in meals, people usually refer to professionals and experts, which is costly and labor-intensive. People also find unreliable information online and are often misinformed about nutritional data and diet fads \cite{williamson2000recommendations}. 




% \begin{figure} [ht]
%   \centering
%   \includegraphics[width=1.0\columnwidth]{figures/decomp2}
%   \caption{Sample picture of ingredients containing carbohydrates selected for the Decomposition strategy group.}~\label{fig:decomp2}
%   \vspace*{-\baselineskip}
% \end{figure}


Recognizing such challenges, there emerged a variety of  interactive solutions for simplifying nutritional assessment of meals captured in the context of diet tracking. Some solutions rely on more structured way of entering nutritional information. For example, many existing diet tracking solutions allow their users to select meals from databases that include meals with known nutritional composition \cite{beijbom2015menu,kong2012dietcam,zhang2015snap,zhu2010use}. Others allow users to scan packaged foods with a barcode reader \cite{siek2009evaluation}. Yet other approaches allow individuals to record their meals in an unstructured way, most commonly by capturing their photograph and textual description, and rely on either crowdsourcing mechanisms \cite{noronha2011platemate} or computational image analysis to arrive at nutritional composition of these meals \cite{anthimopoulos2015computer,beijbom2015menu,kong2012dietcam,rhyner2016carbohydrate,zhang2015snap,zhu2010use}. However, despite ongoing active investigations, there are few easy to use robust solutions that can help reduce the burden of and increase accuracy of arriving at nutritional estimates.



In this research we investigate a different approach to facilitating nutritional literacy and to helping individuals become better at estimating nutrition in meals through leveraging social computing platforms. With the growing popularity of social technologies for diet monitoring and management, there exist vast online collections of meals captured by various individuals across times and geographic locations. Posting one's meals on social media, such as Facebook, Instagram, Yelp, Hipstamatic, or Foodspotting has become a widely adopted practice. Interestingly, these meal photos are often tagged with locations, descriptions, and ingredients. There is an untapped opportunity to utilize these collections of user-generated meal photos and their tagged information to educate the public on nutritional composition of real-life, "in the wild" meals.

In this paper we describe the results of a controlled experiment comparing blah blah blah 

\subsection{Contributions}
In summary, the key contributions of this paper are as follows:

\begin{enumerate}
    \item We take the existing collection of continuous stream data (or time series data) and event-based discrete data (or categorical, discrete-valued time series data) and look at these two data sets from a different angle to produce and discover new insights that can aid in more fine-grained personalization of education.
    \item We implemented this uniquely Human Computer Interaction (HCI) oriented mission in the context of test-taking and test anxiety (represented through heart rate). Through this experience we have come across many technical challenges that were time-consuming and costly. Therefore, we provide suggestions for HCI researchers who may want to replicate this type of research in their own context. (CODE AVAIL ON GITHUB!)
    \item We used k-means to divide students in a way that was different than when we used only one data stream; meaningfully dividing a classroom for better personalization (contribution to the field of education).
    \item We deployed a survey to test whether the k-means tool for group formation aided educators in creating homogeneous groups more efficiently than without the tool. This served as evidence of a useful machine learning educational application utilizing complex sets of data.    
\end{enumerate}


 
